\chapter{Qualité}
\thispagestyle{EIP} % seems mandatory

Notre architecture devra répondre a plusieurs problématiques autre que les contraintes purement fonctionelles.

Comme abordé dans la partie Taille et performance nous avons une charge réseau atypique. Nous n'auront pas a faire face a un grand nombre de requettes, mais chaqune sera potentielement relativement lourde et longue. La solution choisie est de favoriser le plus possible les échanges direct entre les clients et les drivers lorsque c'est possible.

Nous devrons aussi répondre à des exigences des utilisateurs, notament concernant la sécurité de leur donnés et notre capacité à faire face à des pertes de données sur certains drivers ou l'indisponitbilité de ceux-ci. Pour cela nous avons un systeme d'aiguilleur parametrable dont le travail sera de faire en sorte que l'état d'Onitu coresponde a celui décrit dans le fichier de configuration. De cette facon l'utilisateur pour définir par un ensemble de regles le nombre de sauvegarde des fichiers et les priorités des drivers.

Onitu devra également avoir d'autres qualités qui sont attendu d'un systeme semblable. Tout d'abord nous devront assuré la confidentialité des donnés et l'authentification, pour cela nous utiliseront des protocoles déja en place et reconus tel que OAuth.

En ce qui concerne les différents drivers servant à la gestion du stockage de fichiers sur dans  Onitu, la sécurité de leurs protocoles respectifs ne dépend pas de nous. Nous ne pourrons que conseiller l'utilisateur sur la configuration de la solution en fonction de ses attentes, (confidentialité, intégrité, authentification…), par exemple en lui conseillant \textit{SFTP} au lieu de \textit{FTP}.

La fiabilité du systeme est aussi très importantes et sera assuré par une conception simple et modulaire. Si l'un des drivers a un probleme cela n'impactera pas le fonctionement global du systeme et devrais être transparent pour l'utilisateur si des alternatives sont possible, par exemple si nous avons des sauvegardes du fichier demandé sur un autre driver.

La portabilité sera assuré principalement par python, c'est le language principal utilisé dans Onitu et il est présent sur tout les systemes d'exploitation principaux. Nous pretons une attention particuliere aux modules et aux bibliotheques que nous utilisont pour assuré un portage facile.
