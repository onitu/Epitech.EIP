\chapter{Taille et performance}
\thispagestyle{EIP} % seems mandatory

\section{Environnement}

Onitu est destiné à être utilisé avant tout dans un cadre privé par des particuliers. Par conséquent, nous avons des dispositions particulières à prendre car les contraintes ne sont pas les mêmes.

Tout d'abord, les machines sur lesquelles Onitu va être installé seront très diverses et risquent de varier entre entrée de gamme et serveur professionnel. Nous devrons donc prévoir une installation facilitée, et qui s'adapte aux capacités du serveur. Le lancement des différents services et leur administration devront également être simples, de façon à pouvoir être utilisés par n'importe quel utilisateur (amateur ou professionnel).

Bien que nous nous concentrions principalement sur Linux, il nous semble important d'être portable. Nous ferons donc le nécessaire pour que le portage demeure facile lorsque le projet sera plus avancé.

\section{Stockage}

Concernant les performances d'Onitu, elles seront principalement évaluées sur deux points principaux: le stockage et les réponses aux requêtes utilisateur.

En ce qui concerne le stockage, nous devrons faire face à un nombre potentiellement très grand de fichiers. Nous avons donc choisi une architecture simple, qui puisse fonctionner en plusieurs instances. De cette façon, le système est extensible et pourra ainsi fournir un espace de stockage virtuellement illimité, à condition d'avoir les ressources nécessaires.

L'organisation et l'équilibrage interne peuvent se faire sur des périodes de temps plus longues et régulières.

\section{Accès}

Dans un usage typique du produit, nous aurons relativement peu d'utilisateurs et de requêtes simultanés. En effet, une instance Onitu étant principalement destinée à un usage privé ou en entreprise, elle aura un nombre réduit d'utilisateurs, au maximum le nombre d'employés d'une petite entreprise, c'est-à-dire quelques dizaines de personnes.

Ces utilisateurs ne feront pas beaucoup d'opérations simultanées. Nous estimons que nous aurons moins de quelques centaines de requêtes par heure, avec des pics à quelques dizaines. Ces chiffres sont très raisonnables pour un serveur, et ne poseront donc pas de problèmes.

Dans un projet comme Onitu, la difficulté est que les requêtes individuelles peuvent être lourdes, car il s'agira souvent de transfert de fichiers entre un des drivers et le client. Dans la mesure du possible, ces requêtes seront redirigées directement vers le driver concerné, évitant ainsi une surcharge du serveur en façade.

Il est primordial que les requêtes ne soient pas bloquantes et s'effectuent rapidement. Pour cela, nous avons choisi d'effectuer certaines actions de manière asynchrone. Par exemple, la mise à jour des drivers et la redistribution des fichiers en fonction de la charge de chaque plateforme de stockage seront faites en parallèle, indépendamment de quand les requêtes utilisateur sont faites.


