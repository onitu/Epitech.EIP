Pour s'assurer du bon fonctionnement du projet tout au long du développement et pour faciliter les possibles évolutions, des tests sont nécessaires.\\

Des tests sur chacunes des parties du projet doivent donc être créés au fur et à mesure que du code sera écrit. Il est notamment nécessaire de tester :\\

\begin{itemize}
\renewcommand{\labelitemi}{$\bullet$}
\item L'API Ubuntu One, pour s'assurer que toutes les communications avec les clients existants se déroulent correctement ;
\item La bonne copie et l'envoie des fichiers quand ils sont ajoutés dans un dossier qui est synchronisé ;
\item Windows
\end{itemize}

\vspace{0.5cm}

Pour facilier ces tests et être certain qu'ils sont effectués de manière régulière, les tests seront effectués avec un outil d'intégration continue à chaque commit. Les deux outils suivants peuvent être utilisés :\\

\begin{itemize}
\renewcommand{\labelitemi}{$\bullet$}
\item \href{https://travis-ci.org}{travis} qui est un service web lié à Github ;
\item \href{http://jenkins-ci.org}{jenkins} qui est un logiciel libre à installer sur un serveur mais qui permet une beaucoup plus grande variété de tests qu'avec travis.
\end{itemize}
