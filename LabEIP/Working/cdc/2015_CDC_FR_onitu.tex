\documentclass[12pt]{report}

\usepackage[utf8]{inputenc}
\usepackage[french]{babel}
\usepackage{fullpage}
\usepackage{graphicx}
\usepackage{fancyhdr}	% headers/footers
\usepackage{xcolor}		% to use our own color
\usepackage{lastpage}	% to easily know the total number of pages
\usepackage{titling}	% to easily know the total number of pages
\usepackage{colortbl}	% to put color in a table background
\usepackage{datetime}	% to allow us set a new date formatting
\usepackage{multirow}   % to allow multirows in tables

% Custom defines zone

% Define useful hand-made colors
\definecolor{epiBlue}{RGB}{23,54,93}
\definecolor{lightGray}{gray}{0.85}

% Bit of code to bold an entire table row
% http://tex.stackexchange.com/questions/4811/make-first-row-of-table-all-bold
\newcolumntype{$}{>{\global\let\currentrowstyle\relax}}
\newcolumntype{^}{>{\currentrowstyle}}
\newcommand{\rowstyle}[1]{\gdef\currentrowstyle{#1}%
  #1\ignorespaces
}

% Defining a "dd/mm/yyyy" date format
\newdateformat{dashDate}{\twodigit{\THEDAY}/\twodigit{\THEMONTH}/\twodigit{\THEYEAR}}

% Define Document Title
\newcommand{\DocTitle}{[2015][Onitu][CDC]}

% end of Defines


% fancyhdr-specific commands
\setlength{\headheight}{15.2pt}

%% Defining headers and footers contents.

% Big dirty hack of the "empty" pagestyle to show header and footer on the title page (in wait of a better solution)
\fancypagestyle{empty}
{
	\renewcommand{\headrulewidth}{0pt}
	\renewcommand{\footrulewidth}{1pt}
	\fancyhead[L]{\includegraphics[height=42pt]{logo_eip.png}
	}
	\fancyhead[R]{\colorbox{epiBlue}{\color{white}\textbf{\Large{\DocTitle} } }
	}

	\fancyfoot[L]{
   		\textcolor{gray}{\{EPITECH.\}}
	}
	\fancyfoot[C]{
		\jobname
	}
	\fancyfoot[R]{}
}

\fancypagestyle{EIP}
{
	\renewcommand{\headrulewidth}{0pt}
	\renewcommand{\footrulewidth}{1pt}
	\fancyhead[L]{\includegraphics[height=42pt]{logo_eip.png}
	}
	\fancyhead[R]{\colorbox{epiBlue}{\color{white}\textbf{\Large{\DocTitle} } }
	}

	\fancyfoot[L]{
   		\textcolor{gray}{\{EPITECH.\}}
	}
	\fancyfoot[C]{
		\jobname
	}
	\fancyfoot[R]{
		\thepage/\pageref{LastPage}
	}
}

\pagestyle{EIP} % does not seem to work ...

% end of fancyhdr stuff

%Gummi|063|=)

%\title{The Title\\\normalsize A Sub-title}
\title{
	\huge{\textbf{\textcolor{epiBlue}{EIP Onitu} } }\\
	\Large{\textbf{\emph{\textcolor{gray}{Cahier des charges (CDC)} } } }
}


\begin{document}
\addtocontents{toc}{\protect\refstepcounter{page}} % makes the table of contents count pages from 1 (one)
\maketitle

\thispagestyle{empty}
\vspace*{10mm}

\textbf{\emph{\textcolor{epiBlue}{Document synopsis} } }

This document, based on the Onitu project's features, which are detailed in the Cahier des Charges, presents their organization and planification.

It will first report the risks, hypothesis and constraints due to the project building, and the way we'll handle it during the development phase.

Then it presents the features list, representing them on a WBS (Work Breakdown Structure) tree. Those features, grouped in lots, and scaled using milestones described further, are finally positioned on a Gantt chart, which permits a global and synthetic view of the project.

\newpage


\thispagestyle{empty}
\vspace*{10mm}
\textbf{\emph{\textcolor{epiBlue}{Description du document} } }\\

\begin{tabular}{|>{\columncolor[gray]{0.85}\color{epiBlue} \bfseries } l|l|}
\hline
	Titre & \DocTitle\\
\hline
	Date & \dashDate\today \\
\hline
	Auteur & Alexandre BARON\\
\hline
	Responsable & Louis Roché\\
\hline
	E-Mail & onitu\_2015@labeip.epitech.eu\\
\hline
	Sujet & Cahier des charges\\
\hline
	Mots clés & \\
\hline
	Version du modèle & 1.0\\
\hline
\end{tabular}
\vspace*{10mm}

\textbf{\emph{\textcolor{epiBlue}{Tableau des révisions} } }\\


\begin{tabular}{|$l|p{4cm}|p{2cm}|p{5cm}|}
\hline
\rowcolor{lightGray}
\rowstyle{ \color{epiBlue} \bfseries}
	Date & Auteur & Section(s) & Commentaires\\
\hline
	03/03/2013 & Antoine Rozo & Chapitres 1 et 2 & Complétion description du sujet et description de l'environnement \\
\hline
\end{tabular}

\tableofcontents
\addtocontents{toc}{\protect\thispagestyle{empty} 
                    \protect\pagestyle{empty}}
\thispagestyle{empty}

\chapter{Rappel de l'EIP}
\thispagestyle{EIP} % seems mandatory
\setcounter{page}{1} %reset the page count

\section{What is an EIP at Epitech}
Epitech, the European Institute for Technology, is a school in five years. It offers a final major project starting during the student's third year. This is called an EIP: \emph{Epitech Innovative Project}.


The students must organize and form a group of at least five people. They must choose a project that brings new ideas or improves uppon an older project. The EIP is a mandatory and unique passage in an Epitech's student's life because it is so long (months) and the amount of preparation that is required. The goal is to have a marketable product in the end.


\section{Basic principle of the future system}
Onitu is a project aiming to deliver a free, open source implementation of the Ubuntu One server. Ubuntu One is a Canonical (the official Ubuntu sponsor) service permitting to dispose of an online storage space synchronized between multiple computers and compatible devices through a client software. The client and protocol of Ubuntu One are available under free license. However, the server is proprietary and has not been published.


Our goal is to propose a free equivalent of that server: Onitu. It will allow to easily use other hosting services, "in the cloud" or not, in order to extend the available space. For example, the user will be able to use his Dropbox or Amazon S3 account, or FTP servers to store his files, and then synchronize it via Ubuntu One.


The main goal of Onitu is the advanced user, who cares about data centralization problems, and his near circle, to whom he'll make profit the so established server. He doesn't have to be a technical expert, yet rather curious, typically the Ubuntu user profile. Onitu also aims to be used in enterprises who want to easily master the storage of their data.



\chapter{Présentation de l'environnement de réalisation}
\thispagestyle{EIP} % seems mandatory

\section{Environnement de réalisation}
Le logiciel de gestion de version Git est utilisé, afin de profiter d'un système de branches efficace et évitant ainsi les conflits dans le code.
Les dépôts Git sont hébergés sur le site Github, offrant des fonctionnalités annexes telles que les \textit{issues} ou le \textit{Wiki}, proposant ainsi un espace clair pour discuter de différents points du projet, gérer les contribution de la communauté et rédiger la documentation.\\

Aucun environnement de développement n'est imposé aux personnes souhaitant contribuer. Il est donc important que les dépôts restent vides de tous fichiers liés aux éditeurs de textes.\\

La majeure partie du projet est réalisée avec le langage Python, qui est interprété, dynamiquement typé, et reconnu pour sa flexibilité et sa robustesse.\\

Tout le code Python produit doit être en respect avec la PEP8, imposant diverses règles qui amènent à un code propre et agréable à lire, et qui fait office de convention au sein des développeurs.\\

Dans la mesure du possible, le serveur doit être portable sur une majorité de plateformes. Ce travail est facilité par le langage Python, mais nécessite une attention particulière.\\

Afin d'assurer la documentation du projet, nous nous servons du site Read The Docs, et notre documentation est générée avec l'outil Sphinx.
\section{Environnement matériel}
Plusieurs machines seront nécessaires pour tester les différents composants. Ces machines peuvent être virtualisées, car elles nécessiterons peut de ressources dans la plupart des cas.\\

La majorité du temps, les tests peuvent se dérouler localement et ne nécessitent donc pas de structure externe. Cepandent, réguliérement, des essais devront être effectués à échelle réelle, nécessitant plusieurs machines sur des réseaux différents.\\

Les contraintes liés à la portablité entraînent de faire des tests sur différentes plateformes. Des machines fonctionnant sous ces différentes plateformes sont donc requises.\\

Un serveur d'intégration doit être utilisé afin de faire tourner les tests de premier niveaux réguliérement.\\

\section{Architecture technique}
Pour gérer le projet, l'ensemble du travail est regroupé sur une organisation Github : \href{https://github.com/onitu/}{Onitu}.\\

Cette organisation permet notamment d'avoir un dépôt Git pour chacune des parties du projet. Chaque dépôt dispose d'une solution de suivi des problèmes, ce qui permet de surveiller l'avancement du projet et discuter des problèmes.
Pour toutes les étapes du développement, des \textit{milestones} sont créées et des \textit{issues} y sont liées. Cela permet de suivre le bon avancement du travail et de regrouper tout ce qui est lié.\\

Chaque dépôt peut aussi disposer d'un wiki, ce qui permet de mettre à disposition de tous la documentation de chacune des parties, que ce soit la documentation technique à destination de développeurs ou de la documentation utilisateur pour installer ou configurer le logiciel.\\

Toutes les parties du projet sont donc développées séparément les unes des autres. Les connexions entre elles se font grâce aux \textit{submodules} de Git, qui permettent d'intégrer un dépôt Git à l'intérieur d'un autre.\\

Tous les membres du groupe d'EIP ont les droits d'édition sur l'ensemble des dépôts de l'organisation Onitu. Ils peuvent donc participer à chacune des parties du projet, aussi bien sur du développement que sur de la documentation ou des rapports de bogues.\\

L'outil de revue de code intégré à Github est utilisé pour faire valider chaque avancement, dans une optique de qualité et de sécurité.

\section{Composants existants}
\thispagestyle{EIP} % seems mandatory
Nous voulons nous assurer d'être compatibles avec tous les clients Ubuntu One déjà existants, sur toutes les plateformes disponibles (Windows, Mac, Android, iOS).\\

Comme le serveur Ubuntu One d'origine, nous prévoyons également d'utiliser l'implémentation Python des \textbf{Protocol Buffers}, la solution d'organisation de données libre de Google. Nous nous servirons à cet effet du compilateur et de l'API fournis par Google. Nous utiliserons par ailleurs le \emph{« Ubuntu Storage Protocol »}, celui utilisé par Ubuntu One.\\

Nous comptons nous servir, pour la communication réseau, de \textbf{Twisted}, un framework Python de programmation réseau événementielle (\emph{event-driven}). Actuellement dans sa version 12.3, le projet continue d'évoluer. Notamment, écrit à l'origine pour Python 2.x, rendre Twisted entièrement compatible avec Python 3 est un des principaux objectifs des auteurs, ce qui correspond à nos attentes.


\section{Security considerations}

Security has to be taken into accoutn a several stages of software developments. The first thing that comes to mind is to secure the produced software. Often the software is attacked and damage is caused when it in in production. But that is not the only risk. One has to pay the same attention to the development process itself, the handling of the different ressources associated to the project and the project's maintainance.

\subsection{Infrastructure \& maintenance}

The development process is organised around Git and Github. The members of the EIP group have an administrator level access on this platform. This automatically allows them to contribute to the code and use the different tools provided by Github. Git provides a tracability of the code's modifications because each change is signed and authenticated.\\

The source code is hosted on the Github servers, but each developper has a local copy of the repository at all times. This means a data-loss on the hosting platform isn't that much of a problem. There also are no problems associated to confidentiality because the project is open source and there should be nothing private.\\

All the Onitu repositories are open to external contributors, through \textit{pull-request}. This is a request to integrate into the main repository some code developed somewhere else by someone not beeing part of the usual contributors. If \textit{pull-request} is made, the external code shall be audited to make sure it meets Onitu's quality and security standarts. It will be checked that the code doesn't introduce any new vulnerabilities or takes away functionalities.\\

User feed-back will be possible through Github's issue management, its ticket system will allow for easy management of bugs and security issues. Anyone can create tickets but they can only be closed, deleted or administrated by the members of the EIP group.

\subsection{Development processus}

Almost the whole project will be developed using the Python language. This eliminates a big part of the clasic vulnerabilities one encounters in projects developed with lower-level languages, in particular those concerning memory menagement which is no longer left to the programer. Furthermore Python raises exceptions and stops the program as soon a value is not used the way it should. (\textit{overflows}, incorect \textit{casts}…)\\

The interactions with the database and the different protocols are made throufh libraries offering easy usage and which are less error-prone than a direct access to the ressources.\\

However, a certain number of vulnerabilities can only be avoided by using common sense and respect of best-practises. This is in particular the case for everything concerning function allowing code evaluation or access to system ressources. Calling this functions shall therefor have to be justified and supervised, even more so when they can be influenced by user input.\\

Auditing sessions shall be organised where the whole code base shall be checked by developpers who didnt work on those parts. Doing thins this way we hope to have a second, maybe more objectif, point of view. Furthermore we have a agreement with Epitech Toulouse's security lab which will be able to provide one or more external audits on the code and the different protocols that are beeing used.

\subsection{Security in Onitu}

In addition to the different mesures explained above, a part of the security of Onitu comes from its main components and the protocols beeing used in them.

\subsubsection{Ubuntu One}
Beeing compatible with Ubuntu One, Onitu can't change the protocols in any way. However those protocols are designed and implemented by profressional developers and are used in production for several years now.\\

Like Ubuntu One, Onitu uses \textit{SSL} (Secure Socket Layer) and numeric certificates to ensure authentication of the server and encryption of the transactions.\\

To authenticate cliens, the protocol \textit{OAuth} is used. It is a \textit{token}-based authentication system that allows managing parral access from diferent computers and their revocations if necessary.\\

The stored files are not encrypted, because the serveur needs the original version when a download request occures. However nothing prevents the user to cypher its files localy before syncronising them with Onitu's server.

\subsubsection{Backend drivers}
Concerning the different driver modules necessary for the backend storage facilities on the Onitu server, the security of their different protocols can't be influenced by Onitu's developpers. We can only advice the user on the configuration of the solution considering his needs, (confidentiality, integrity, authentication…), for exemple we could advice \textit{SFTP} instead of \textit{FTP}.

\section{Sensibles points}
Onitu has to be finished in two and a half years. During this period several sensible points must be payed particular attention to.\\

A first important point is Ubuntu One's protocol. Onitu might have to adapt quickly if Canoncical decides to change parts of its protocol. However such a decision is very unlickly and is unexpected.\\

Portability is another subject of special attention. This means we want to run as may tests as we can on a great number of platforms but that we might we obligated to change a library or functionality because it is not suported everywhere. IF a plateform cause to many troubles in isn't widly used we might consider dropping its support.\\

Onitu is confronted to the \textit{CAP} theorem, stating that in practical computer science it is impossible for a distributed computer system to simultaneously provide consitency, availability and partiion tolerance. Onitu's priority is preserving consitency and availability.\\

In general, development of a server is always a delicate matter that should not be under-estimated. Numerous test in different categories must we undertaken regulary and user feedback must be taken into account.


\chapter{Description des différentes parties du programme à réaliser}
\thispagestyle{EIP} % seems mandatory

\chapter{Description de la base de données}
\thispagestyle{EIP} % seems mandatory

\chapter{Description des tests de premier niveau}
\thispagestyle{EIP} % seems mandatory
Pour s'assurer du bon fonctionnement du projet tout au long du développement et pour faciliter les possibles évolutions, des tests sont nécessaires.\\

Des tests sur chacunes des parties du projet doivent donc être créés au fur et à mesure que du code sera écrit. Il est notamment nécessaire de tester :\\

\begin{itemize}
\renewcommand{\labelitemi}{$\bullet$}
\item L'API Ubuntu One, pour s'assurer que toutes les communications avec les clients existants se déroulent correctement ;
\item La bonne copie et l'envoie des fichiers quand ils sont ajoutés dans un dossier qui est synchronisé ;
\item Windows
\end{itemize}

\vspace{0.5cm}

Pour facilier ces tests et être certain qu'ils sont effectués de manière régulière, les tests seront effectués avec un outil d'intégration continue à chaque commit. Les deux outils suivants peuvent être utilisés :\\

\begin{itemize}
\renewcommand{\labelitemi}{$\bullet$}
\item \href{https://travis-ci.org}{travis} qui est un service web lié à Github ;
\item \href{http://jenkins-ci.org}{jenkins} qui est un logiciel libre à installer sur un serveur mais qui permet une beaucoup plus grande variété de tests qu'avec travis.
\end{itemize}


\chapter{Organisation projet}
\thispagestyle{EIP} % seems mandatory
Nous avons la chance de débuter le projet dans des conditions proches de ce que nous vivrons l'année prochaine, au décalage horaire prêt, dans le sens où nous sommes répartis sur trois sites différents. Ainsi, notre organisation prend en compte cet élément dès le départ, et nous ne subirons pas de choc, ou moindre, l'an prochain.\\

Pour cela, nous avons au début du projet mis en place un système de réunions hebdomadaires, à savoir chaque mercredi soir, où nous discutons principalement des documents à rendre ainsi que de leurs échéances. En période de crise, il nous arrive aussi de prévoir de nouvelles réunions.\\

En parallèle, nous sommes constamment présents sur un channel IRC sur lequel nous débattons des axes que nous souhaitons prendre pour notre projet. Il nous sert aussi aux préparatifs des réunions, c'est en effet là que nous fixons les heures et nous réunissons préalablement, après quoi nous passons sur le système d'Hangouts de Google pour procéder à la réunion proprement dite.\\

De plus, notre projet étant hébergé sur Github, nous bénéficions des avantages de ce dernier dans sa gestion, à savoir principalement un wiki sur lequel nous écrivons par exemple nos compte-rendus de réunions ou mettons à disposition des conversations importantes que nous avons pu avoir, ainsi qu'un système d'issues et milestones, qui nous servent à la répartition des tâches, et à avoir une vision des échéances.

\chapter{Annexes}
\thispagestyle{EIP} % seems mandatory

\end{document}
