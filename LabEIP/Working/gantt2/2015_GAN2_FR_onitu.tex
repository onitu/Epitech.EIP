\documentclass[11pt]{report}

\usepackage[utf8]{inputenc}
\usepackage[french]{babel}
\usepackage{fullpage}
\usepackage{graphicx}
\usepackage{fancyhdr}	% headers/footers
\usepackage{xcolor}		% to use our own color
\usepackage{lastpage}	% to easily know the total number of pages
\usepackage{titling}	% to easily know the total number of pages
\usepackage{colortbl}	% to put color in a table background
\usepackage{datetime}	% to allow us set a new date formatting
\usepackage{multirow}   % to allow multirows in tables
\usepackage{pgfgantt}
\usepackage{lscape}
\usepackage[colorlinks,linkcolor=black]{hyperref}
\usepackage{palatino}
%% \usepackage[colorlinks=false, urlcolor=blue, breaklinks, pagebackref, citebordercolor={0 0 0}, filebordercolor={0 0 0}, linkbordercolor={0 0 0}, pagebordercolor={0 0 0},
%%                      runbordercolor={0 0 0}, urlbordercolor={0 0 0}, pdfborder={0 0 0}]{hyperref}

% Custom defines zone

% Define useful hand-made colors
\definecolor{epiBlue}{RGB}{0,110,255}
\definecolor{lightGray}{gray}{0.92}

% Bit of code to bold an entire table row
% http://tex.stackexchange.com/questions/4811/make-first-row-of-table-all-bold
\newcolumntype{$}{>{\global\let\currentrowstyle\relax}}
\newcolumntype{^}{>{\currentrowstyle}}
\newcommand{\rowstyle}[1]{\gdef\currentrowstyle{#1}%
  #1\ignorespaces
}

% Defining a "dd/mm/yyyy" date format
\newdateformat{dashDate}{\twodigit{\THEDAY}/\twodigit{\THEMONTH}/\twodigit{\THEYEAR}}

% Define Document Title
\newcommand{\ProjectTitle}{Onitu}
\newcommand{\DocTitle}{Gantt 2}
\newcommand{\SubTitle}{Diagramme de Gantt et WBS}

% Defining some logo image names
\newcommand{\ProjectLogo}{}
\newcommand{\EIPLogo}{logo_eip.png}

% Setting the space between each page's header and its content
\setlength{\headsep}{0.2in} 

% end of Defines


% fancyhdr-specific commands
\setlength{\headheight}{15.2pt}

%% Defining headers and footers contents.

% Big dirty hack of the "empty" pagestyle to show header and footer on the title page (in wait of a better solution)
\fancypagestyle{empty}
{
	\renewcommand{\headrulewidth}{0pt}
	\renewcommand{\footrulewidth}{1pt}
	\fancyhead[L]{\includegraphics[height=35pt]{\EIPLogo}}
	\fancyhead[C]{}
	\fancyhead[R]{\textcolor{epiBlue}{\textbf{\emph{\huge{\ProjectTitle}}}}}

	\fancyfoot[L]{}
	\fancyfoot[C]{\textcolor{epiBlue}{\textbf{\underline{\DocTitle\ — \SubTitle}}}}
	\fancyfoot[R]{}
}

\fancypagestyle{EIP}
{
	\renewcommand{\headrulewidth}{0pt}
	\renewcommand{\footrulewidth}{1pt}
	\fancyhead[L]{\includegraphics[height=35pt]{\EIPLogo}}
	\fancyhead[C]{}
	\fancyhead[R]{\textcolor{epiBlue}{\textbf{\emph{\huge{Onitu}}}}}

	\fancyfoot[L]{\textcolor{epiBlue}{\textbf{\underline{\leftmark}}}}
	\fancyfoot[C]{}
	\fancyfoot[R]{
		\thepage/\pageref{LastPage}
	}
}

\pagestyle{EIP} % does not seem to work ...

% end of fancyhdr stuff

%Gummi|063|=)

%\title{The Title\\\normalsize A Sub-title}
\title{
	\huge{\textbf{\textcolor{epiBlue}{\DocTitle} } }\\
	\Large{\textbf{\emph{\textcolor{gray}{\SubTitle} } } }
}


\begin{document}
\addtocontents{toc}{\protect\refstepcounter{page}} % makes the table of contents count pages from 1 (one)
\maketitle

\thispagestyle{empty}
\vspace*{10mm}

\textbf{\emph{\textcolor{epiBlue}{Document synopsis} } }

This document, based on the Onitu project's features, which are detailed in the Cahier des Charges, presents their organization and planification.

It will first report the risks, hypothesis and constraints due to the project building, and the way we'll handle it during the development phase.

Then it presents the features list, representing them on a WBS (Work Breakdown Structure) tree. Those features, grouped in lots, and scaled using milestones described further, are finally positioned on a Gantt chart, which permits a global and synthetic view of the project.

\newpage


\thispagestyle{empty}
\vspace*{10mm}
\textbf{\emph{\textcolor{epiBlue}{\large{Description du documentstash stash} } } } \\

\vspace*{2mm}

\begin{tabular}{|>{\columncolor{epiBlue} \color{lightGray} \bfseries } l|l|}
\hline
	Titre & \DocTitle\\
\hline
	Date & \dashDate\today \\
\hline
	Auteur & Louis Roché\\
\hline
	Responsable & Louis Roché\\
\hline
	E-Mail & onitu\_2015@labeip.epitech.eu\\
\hline
	Sujet & \DocTitle\\
\hline
	Mots clés & Gantt, WBS\\
\hline
	Version du modèle & 2.1\\
\hline
\end{tabular}

\vspace*{10mm}

\textbf{\emph{\textcolor{epiBlue}{\large{Tableau des révisions} } } }\\

\vspace*{2mm}

\begin{tabular}{|$l|p{4cm}|p{2cm}|p{5cm}|}
\hline
\rowcolor{epiBlue}
\rowstyle{ \color{lightGray} \bfseries}
	Date & \textcolor{lightGray}{\textbf{Auteur}} & \textcolor{lightGray}{\textbf{Section(s)}} & \textcolor{lightGray}{\textbf{Commentaires}}\\
\hline
        06/05/2013 & Louis Roché & Gantt & Ajouts des etapes du labeip sur le gantt \\
\hline
        15/05/2013 & Louis Roché & Gantt & Ajouts des charges de travail \\
\hline
	& & & \\
\hline
\end{tabular}

\tableofcontents
\addtocontents{toc}{\protect\thispagestyle{empty}
                    \protect\pagestyle{empty}}
\thispagestyle{empty}

\chapter{Rappel de l'EIP}
\thispagestyle{EIP} % seems mandatory
\setcounter{page}{1} %reset the page count

\section{What is an EIP at Epitech}
Epitech, the European Institute for Technology, is a school in five years. It offers a final major project starting during the student's third year. This is called an EIP: \emph{Epitech Innovative Project}.


The students must organize and form a group of at least five people. They must choose a project that brings new ideas or improves uppon an older project. The EIP is a mandatory and unique passage in an Epitech's student's life because it is so long (months) and the amount of preparation that is required. The goal is to have a marketable product in the end.


\section{Basic principle of the future system}
Onitu is a project aiming to deliver a free, open source implementation of the Ubuntu One server. Ubuntu One is a Canonical (the official Ubuntu sponsor) service permitting to dispose of an online storage space synchronized between multiple computers and compatible devices through a client software. The client and protocol of Ubuntu One are available under free license. However, the server is proprietary and has not been published.


Our goal is to propose a free equivalent of that server: Onitu. It will allow to easily use other hosting services, "in the cloud" or not, in order to extend the available space. For example, the user will be able to use his Dropbox or Amazon S3 account, or FTP servers to store his files, and then synchronize it via Ubuntu One.


The main goal of Onitu is the advanced user, who cares about data centralization problems, and his near circle, to whom he'll make profit the so established server. He doesn't have to be a technical expert, yet rather curious, typically the Ubuntu user profile. Onitu also aims to be used in enterprises who want to easily master the storage of their data.



\chapter{Contexte}
\section{Risques}
Notre projet étant dans le domaine du cloud qui est très à la mode ces dernieres années, et qui voit naitre de nouvelles solutions régulierement, il y'as un risque qu'une solution équivalent concurente plus avancée voit le jour, mais nous suposons pour l'instant que ce ne sera pas le cas.

\section{Hypothèses}
Nous suposons que les developpeurs travaillant déjà sur les diférents systèmes par lesquels notre projet est concerné sont ouvert et disposé a nous aiguiller. D'après nos premiers échanges, nous suposons aussi qu'ils ne sont pas hostile a notre implémentation open-source.

Nous avons l'intention de ne développer qu'un serveur et donc de nous reposer sur des clients ubuntu-one déjà existants pour nos tests. Ces clients devront très probablement être modifiés legerement de facon a pouvoir les faire fonctionner avec d'autres serveurs que ceux de canonical. Nous esperont que ces modification seront simple et bien recu par leur auteurs respectifs.

Nous ésperont pouvoir nous organiser de facon de facon éfficace durant la quatrieme année en fesant des groupres de travail par fuseau horaire ce qui devrasis augmenter notre productivité et tout de meme permettre une bonne avancée de projet durant l'année a venir.

\section{Contraintes}
Le projet étant opensource nous devrons respecter certaines contraintes imposés par les licenses choisis. Celle-ci doivent elle mêmes encore être étudier et determiner.

L'architecture complete du projet devra être fixer avant Juin 2013 pour pouvoir être presenter durant la Soutenance avec le LabEIP.

Une des contraintes principales de notre projet sera inhérente au dévelopement de celui-ci. En effet il faudra tenir compte du besoin de modification des clients et de l'implementation d'au moins certaines interfaces de stockage (par exemple le stockage local) avant de pouvoir réelement essayer nos idées dans des conditions réeles.

Nous devrons aussi être capable d'organiser des réunions régulieres, et ce malgrès le décalage horraire.

Enfin, du à l'eloigenement géographique il sera plus compliquer de maintenir une platforme de test pour notre solution d'hebergement. Nous seront probablement ammené a prendre un serveur dédiée de test accesible de nos diférentes destinations en tek4.


\thispagestyle{EIP}

\chapter{Planification}
\thispagestyle{EIP}
\section{WBS}
\newpage
\begin{figure}[ht]
    \includegraphics[width=\textwidth,height=\textheight,keepaspectratio]{wbs.png}
    \caption{Work Breakdown Structure}
\end{figure}


\section{Jalons principaux du projet et lotissement}
\subsection{Jalons du planning EIP Epitech}
Cet ensemble de jalons est commun à tous les groupes, car imposé par le planning EIP.


Une première famille de jalons est formée par les bilans d'architecture AA1 et AA2, respectivement du 20 juin et 1er septembre 2013. Ces jalons devront donc mettre terme aux choix et possibilités au niveau de l'architecture du projet.

Une seconde concerne les bilans techniques, et bilans techniques finaux, répartis de façon régulière de novembre 2013 à janvier 2015. Ils s'axeront avec des jalons propres à notre projet et devront marquer des tournants décisifs, tels que des finalisations de lots.


Enfin, la troisième rassemble les deux soutenances finales, de 4ème année en septembre 2014, et de 5ème en février 2015. Ils représenteront pour leur part des livraisons de versions avancée pour la première, et finale pour la seconde.


\subsection{Jalons propres au projet Onitu}
Ces jalons sont définis au niveau de notre projet et marquent donc des points importants de celui-ci. Ils sont à mettre en parallèle avec l'ensemble précédent, dans le sens où ces derniers représentaient des dates importantes, pour lesquelles ces points devront être atteints.


Tout d'abord, il sera nécessaire de posséder un client minimal de test, pour permettre le développement du serveur, tout en continuant celui du client.


Ensuite, l'obtention d'un serveur fonctionnel, puis d'un serveur complet et fonctionnel marqueront deux étapes très importantes du projet.


Les derniers jalons correspondent à la finalisation de l'interface utilisateur et des drivers minimum prévus pour fonctionner avec Onitu.


\subsection{Définition du lotissement}
Notre lotissement est formé des différentes parties du projet, à savoir: le serveur (conception et réalisation des API et de la base de données), l'interface utilisateur (web et Qt; interface graphique complète de gestion des services et des fichiers), et les drivers (donnant accès à des services internes comme externes pour le stockage et la synchronisation des données).

\section{Gantt}

%\chapter{Annexes}
%\thispagestyle{EIP}

\end{document}

% client web, drivers, serveur, marketing, supervision (rendu, gestion), tests/review (wannes, yannick, antoine)
