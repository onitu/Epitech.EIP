\section{Risques}
Notre projet étant dans le domaine du cloud qui est très à la mode ces dernieres années, et qui voit naitre de nouvelles solutions régulierement, il y'as un risque qu'une solution équivalent concurente plus avancée voit le jour, mais nous suposons pour l'instant que ce ne sera pas le cas.

\section{Hypothèses}
Nous suposons que les developpeurs travaillant déjà sur les diférents systèmes par lesquels notre projet est concerné sont ouvert et disposé a nous aiguiller. D'après nos premiers échanges, nous suposons aussi qu'ils ne sont pas hostile a notre implémentation open-source.

Nous avons l'intention de ne développer qu'un serveur et donc de nous reposer sur des clients ubuntu-one déjà existants pour nos tests. Ces clients devront très probablement être modifiés legerement de facon a pouvoir les faire fonctionner avec d'autres serveurs que ceux de canonical. Nous esperont que ces modification seront simple et bien recu par leur auteurs respectifs.

Nous ésperont pouvoir nous organiser de facon de facon éfficace durant la quatrieme année en fesant des groupres de travail par fuseau horaire ce qui devrasis augmenter notre productivité et tout de meme permettre une bonne avancée de projet durant l'année a venir.

\section{Contraintes}
Le projet étant opensource nous devrons respecter certaines contraintes imposés par les licenses choisis. Celle-ci doivent elle mêmes encore être étudier et determiner.

L'architecture complete du projet devra être fixer avant Juin 2013 pour pouvoir être presenter durant la Soutenance avec le LabEIP.

Une des contraintes principales de notre projet sera inhérente au dévelopement de celui-ci. En effet il faudra tenir compte du besoin de modification des clients et de l'implementation d'au moins certaines interfaces de stockage (par exemple le stockage local) avant de pouvoir réelement essayer nos idées dans des conditions réeles.

Nous devrons aussi être capable d'organiser des réunions régulieres, et ce malgrès le décalage horraire.

Enfin, du à l'eloigenement géographique il sera plus compliquer de maintenir une platforme de test pour notre solution d'hebergement. Nous seront probablement ammené a prendre un serveur dédiée de test accesible de nos diférentes destinations en tek4.

