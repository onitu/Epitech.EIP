\chapter{iCloud}
\thispagestyle{EIP} % seems mandatory

\section{Présentation}
Principalement utile pour les utilisateurs de produits Apple, iCloud permet une synchronisation, par cloud, entre les différents appareils qu'elle produit, il est par exemple fourni avec iOS.

\section{Historique}
Créée en 2011, cette application de la firme Apple regroupe iTunes in the Cloud, Photo stream, Calendar, Mail, et Contacts.

\section{Description}
Cette application permet la synchronisation de différents éléments (applications, livres, documents, sauvegardes) entre divers appareils. Il ne s'agit pas à proprement parler d'un service par navigateur, maisil dispose d'une interface web pour contrôler certaines de ses données.\\
Ses utilisateurs peuvent partager leurs fichiers à l'aide de la suite iWork, et bénéficient d'un utilitaire de sauvegarde automatique.\\
\\
Ils ont pour cela un espace allant de 5 à 55Go à leur disposition, ainsi qu'une taille maximale de 250Mo par fichier.\\
Il est intégré à iOS et propose des applications pour Windows et Mac OS.\\

\section{Critiques}
Cette application n'est disponible que pour très peu de systèmes, et l'interface web fortement limitée (accès uniquement aux données d'iWork). Pour en profiter pleinement, l'utilisateur doit ainsi s'orienter vers du tout Apple.