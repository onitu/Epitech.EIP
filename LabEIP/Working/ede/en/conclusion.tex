We begin with a with a table summarizing the different projects we talk about in this document. We also note the points that we think contribute the most to a good user experience.\\

\begin{tabular}{|$l|p{4cm}|p{4cm}|p{5cm}|}
\hline
\rowcolor{lightGray}
\rowstyle{ \color{epiBlue} \bfseries}
	Projet & Licence & Independance & Platform \\
\hline
	Dropbox & Proprietary & No & GNU/Linux, Windows, Mac OS, Blackberry, iOS, Android \\
\hline
	Sparkleshare & GPL3 & Yes & GNU/Linux, Windows, Mac OS, iOS, Android \\
\hline
	Google drive & Proprietary & No & Web \\
\hline
	Skydrive & Proprietary & No & Windows, Mac OS, iOS, Web \\
\hline
	Owncloud & AGPL & Auto-hébergement & All \\
\hline
	iCloud & Proprietary & No & Windows, Mac OS, iOS \\
\hline
	Syncany & GPL3 & Yes & All \\
\hline
	Ubuntu One & Proprietary Server & No & GNU/Linux, Windows, Mac OS, iOS, Android \\
\hline
	Onitu & Libre & Auto-hébergement & GNU/Linux, Windows, Mac OS, iOS, Android \\
\hline
\end{tabular}

\vspace*{10mm}

Our projects strengthens Ubuntu One because it allows for one to host the server whererver one wants. While maintaining its main advantages (Existing open-source and multi-platform clients, protobuffer based protocols, etc) Onitu addresses UbuntuOne's main disadvantage: Its closed-source and centralized server.

\vspace*{10mm}

SWOT
