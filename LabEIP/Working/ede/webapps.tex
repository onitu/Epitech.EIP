\chapter{Services de cloud par navigateur}
\thispagestyle{EIP} % seems mandatory

\section{Présentation}
Ces trois services sont assez similaires. notamment par rapport au fait qu'ils offrent une interface de gestion accessible depuis un navigateur internet, et sont donc compatibles tous systèmes. Cela ne les empêche pas de proposer des applications spécifiques pour différentes plate-formes.


\section{Google drive}

\subsection{Historique}
Google Documents est une application web créé en 2006, regroupant un tableur, un traitement de textes et un éditeur de diaporamas.
Lancé en 2012, Google Drive lui succède, proposant en plus un système de stockage en ligne, valable pour un grand nombre de types de fichiers.\\

\subsection{Description}
Ce service hérite des fonctionnalités de Google Documents, telles que le partage de données avec d'autres utilisateurs, mais aussi, par son interface, une édition simultanée d'un même document par plusieurs personnes (travail collaboratif).\\
Le principal apport de Google Drive est certainement la possibilité de synchroniser ses données avec des fichiers locaux, qui fait de lui un véritable service de cloud computing.\\
Il s'intègre parfaitement à d'autres applications Google comme Google+ ou Gmail.\\
\\
Il offre des capacités allant de 5Go à 16To, tout en permettant 10Go maximum par fichier, et possède des applications spécifiques pour les plate-formes Windows, Mac OS, Android, et prochainement iOS.\\


\section{iCloud}

\subsection{Historique}
Créée en 2011, cette application de la firme Apple regroupe iTunes in the Cloud, Photo stream, Calendar, Mail, et Contacts.

\subsection{Description}
Cette application permet la synchronisation de différents éléments (applications, livres, documents, sauvegardes) entre divers appareils. Il ne s'agit pas à proprement parler d'un service par navigateur, maisil dispose d'une interface web pour contrôler certaines de ses données.\\
Ses utilisateurs peuvent partager leurs fichiers à l'aide de la suite iWork, et bénéficient d'un utilitaire de sauvegarde automatique.\\
\\
Ils ont pour cela un espace allant de 5 à 55Go à leur disposition, ainsi qu'une taille maximale de 250Mo par fichier.\\
Il est intégré à iOS et propose des applications pour Windows et Mac OS.\\


\section{Windows Live Skydrive}

\subsection{Historique}
Windows Live Folders est lancé puis ouvert au grand public en août 2007. Il change de nom le même mois pour devenir Window Live Skydrive. Offrant au départ un espace de stockage gratuit de 5Go par utilisateur, il passe à 25Go en 2008 pour redescendre à 7Go en 2012.\\

\subsection{Description}
Il peut s'utiliser aussi bien en tant que service web qu'en tant qu'application lourde, permettant une synchronisation avec les données locales.\\
Un service de travail collaboratif est aussi disponible de façon à ce que différents utilisateurs puissent travailler sur les mêmes fichiers.\\
Il permet un accès à distance aux postes ayant installé le client Windows Live Skydrive.\\
\\
Ses capacités varient de 7 à 107Go, avec 2Go maximum par fichier.\\
Inclu à Office et Windows Phone, il possède des applications pour Windows, Mac OS, iOS et bientôt pour Android.\\


\section{Critiques}
Bien que leurs interfaces respectives puissent être très agréables à utiliser, ces trois services souffrent d'un même problème: une centralisation des données qui échappe à l'utilisateur. Ce dernier n'a effectivement aucun contrôle sur ce qu'il stocke, et le code fermé de ces applications empêche de les porter sur d'autres sites.