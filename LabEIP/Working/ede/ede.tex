\documentclass[12pt]{report}

\usepackage[utf8]{inputenc}
\usepackage[french]{babel}
\usepackage{fullpage}
\usepackage{graphicx}
\usepackage{fancyhdr}	% headers/footers
\usepackage{xcolor}		% to use our own color
\usepackage{lastpage}	% to easily know the total number of pages
\usepackage{titling}	% to easily know the total number of pages
\usepackage{colortbl}	% to put color in a table background
\usepackage{datetime}	% to allow us set a new date formatting

% Custom defines zone

% Define useful hand-made colors
\definecolor{epiBlue}{RGB}{23,54,93}
\definecolor{lightGray}{gray}{0.85}

% Bit of code to bold an entire table row
% http://tex.stackexchange.com/questions/4811/make-first-row-of-table-all-bold
\newcolumntype{$}{>{\global\let\currentrowstyle\relax}}
\newcolumntype{^}{>{\currentrowstyle}}
\newcommand{\rowstyle}[1]{\gdef\currentrowstyle{#1}%
  #1\ignorespaces
}

% Defining a "dd/mm/yyyy" date format
\newdateformat{dashDate}{\twodigit{\THEDAY}/\twodigit{\THEMONTH}/\twodigit{\THEYEAR}}

% Define Document Title
\newcommand{\DocTitle}{[2015][UbuntuOne][EDE]}

% end of Defines


% fancyhdr-specific commands
\setlength{\headheight}{15.2pt}

%% Defining headers and footers contents.

% Big dirty hack of the "empty" pagestyle to show header and footer on the title page (in wait of a better solution)
\fancypagestyle{empty}
{
	\renewcommand{\headrulewidth}{0pt}
	\renewcommand{\footrulewidth}{1pt}
	\fancyhead[L]{\includegraphics[height=42pt]{logo_eip.png}
	}
	\fancyhead[R]{\colorbox{epiBlue}{\color{white}\textbf{\Large{\DocTitle} } }
	}

	\fancyfoot[L]{
   		\textcolor{gray}{\{EPITECH.\}}
	}
	\fancyfoot[C]{
		\jobname
	}
	\fancyfoot[R]{}
}

\fancypagestyle{EIP}
{
	\renewcommand{\headrulewidth}{0pt}
	\renewcommand{\footrulewidth}{1pt}
	\fancyhead[L]{\includegraphics[height=42pt]{logo_eip.png}
	}
	\fancyhead[R]{\colorbox{epiBlue}{\color{white}\textbf{\Large{\DocTitle} } }
	}

	\fancyfoot[L]{
   		\textcolor{gray}{\{EPITECH.\}}
	}
	\fancyfoot[C]{
		\jobname
	}
	\fancyfoot[R]{
		\thepage/\pageref{LastPage}
	}
}

\pagestyle{EIP} % does not seem to work ...

% end of fancyhdr stuff

%Gummi|063|=)

%\title{The Title\\\normalsize A Sub-title}
\title{
	\huge{\textbf{\textcolor{epiBlue}{EIP Ubuntu One} } }\\
	\Large{\textbf{\emph{\textcolor{gray}{Étude de l'existant (EDE)} } } }
}


\begin{document}
\addtocontents{toc}{\protect\refstepcounter{page}} % makes the table of contents count pages from 1 (one)
\maketitle

\thispagestyle{empty}
\vspace*{10mm}

\textbf{\emph{\textcolor{epiBlue}{Document synopsis} } }

This document, based on the Onitu project's features, which are detailed in the Cahier des Charges, presents their organization and planification.

It will first report the risks, hypothesis and constraints due to the project building, and the way we'll handle it during the development phase.

Then it presents the features list, representing them on a WBS (Work Breakdown Structure) tree. Those features, grouped in lots, and scaled using milestones described further, are finally positioned on a Gantt chart, which permits a global and synthetic view of the project.

\newpage


\thispagestyle{empty}
\vspace*{10mm}
\textbf{\emph{\textcolor{epiBlue}{Description du document} } }\\

\begin{tabular}{|>{\columncolor[gray]{0.85}\color{epiBlue} \bfseries } l|l|}
\hline
	Titre & \DocTitle\\
\hline
	Date & \dashDate\today \\
\hline
	Auteur & Alexandre BARON\\
\hline
	Responsable & Louis Roché\\
\hline
	E-Mail & serveurubintu\_un2015@labeip.epitech.eu\\
\hline
	Sujet & Étude de l'existant autour de notre projet\\
\hline
	Mots clés & Étude, état de l'art, cloud\\
\hline
	Version du modèle & 1.0\\
\hline
\end{tabular}
\vspace*{10mm}

\textbf{\emph{\textcolor{epiBlue}{Tableau des révisions} } }\\


\begin{tabular}{|$l|p{4cm}|p{2cm}|p{5cm}|}
\hline
\rowcolor{lightGray}
\rowstyle{ \color{epiBlue} \bfseries}
	Date & Auteur & Section(s) & Commentaires\\
\hline
	05/02/2013 & Alexandre BARON & Toutes & Première version \\
\hline
	12/02/2013 & Alexandre BARON & Chapitre 1 & Remplissage du Rappel sur l'EIP \\
\hline
	12/02/2013 & Wannes ROMBOUTS & Chapitre 2 & Ajout de l'étude sur Dropbox \\
\hline
	12/02/2013 & Maxime CONSTANTINIAN & Chapitre 3 & Ajout de l'étude sur SparkleShare \\
\hline
	12/02/1013 & Antoine ROZO & Chapitre 4 & Ajout de l'étude sur Google Drive, iCloud et SkyDrive \\
\hline
	12/02/2013 & Louis ROCHÉ & Chapitre 5 & Ajout de l'étude sur Syncany \\
\hline
	12/02/2013 & Yannick PÉROUX & Chapitre 6 & Ajout de l'étude sur OwnCloud \\
\hline
	12/02/2013 & Louis ROCHÉ & Chapitres 5 et 6 & Corrections mineures \\
\hline
	12/02/2013 & Yannick PÉROUX & Chapitre 3 & Améliorations de l'étude sur SparkleShare \\
\hline
	& & & \\
\hline
\end{tabular}

\tableofcontents
\thispagestyle{empty}

\chapter{Rappel de l'EIP}
\setcounter{page}{1} %reset the page count

\section{Qu'est-ce qu'un EIP et Epitech}
Epitech, école d'expertise informatique en cinq ans, propose aux étudiants, à partir de leur troisième année, un projet de fin d'études: l'EIP (pour \emph{Epitech Innovative Project}).\\

À ce titre, les élèves doivent s'organiser en un groupe d'au moins cinq personnes et choisir un sujet porteur de nouveautés ou améliorant un ancien sujet. L'EIP est un passage obligatoire et unique dans la scolarité de l'étudiant, de par son envergure (18 mois) et la préparation requise. Le but est, à la fin du temps imparti, d'obtenir un projet commercialisable.


\section{Sujet de notre EIP}
    Il s’agit d’un projet visant à proposer une implémentation libre et Open Source du serveur d’Ubuntu One.\\

    Ubuntu One est un service de Canonical (sponsor officiel d'Ubuntu) permettant de disposer d’un espace de stockage en ligne qui sera synchronisé entre différents ordinateurs et périphériques compatibles via un logiciel client. Le client et le protocole d’Ubuntu One sont disponibles sous licence libre. Néanmoins, le serveur est propriétaire et n’a pas été publié.\\


    Notre objectif est donc de proposer un équivalent libre de ce serveur, afin
    de profiter des fonctionnalités d’Ubuntu One par exemple à des fins d’autohébergement.\\

    Le client officiel n’étant pas capable d’utiliser un serveur différent de celui
    de Canonical, nous prévoyons également d'effectuer un fork afin d'ajouter cette option.

\thispagestyle{EIP} % seems mandatory

\section{Dropbox}
\thispagestyle{EIP} % seems mandatory
\subsection{Présentation}
DropBox est un service de stockage dans le cloud qui permet la synchronisation de fichiers entre différents terminaux. Dropbox a longtemps été la solution de référence et est énormément utilisé. Il est compatible avec GNU/Linux, Windows, Mac, Blackberrry, iOS et Android.\\

\subsection{Historique}
Le projet Dropbox est né au MIT en 2007 et a été lancé un an plus tard. En 2011, OPSWAT rapporte que Dropbox représente 14.14\% du marché mondial et cette même année, il dépasse les 50 millions d'utilisateurs. En 2012, ce chiffre est doublé et Dropbox annonce 100 millions d'utilisateurs.\\

\subsection{Description}
Dropbox crée un dossier spécial sur chaque ordinateur où il est installé. Il va synchroniser ce dossier entre les différents terminaux en répercutant les modifications apportées aux fichiers ou sous-dossiers. Les données placées dans ce dossier sont aussi accessibles depuis une interface web.\\

Les utilisateurs de Dropbox ont gratuitement accès à un espace de stockage de 18 Go, avec la possibilité de prendre un compte Pro pour bénéficier d'espace supplémentaire en payant un abonnement mensuel.\\

D'un point de vue technique, le serveur et le client Dropbox sont tous deux écrits en Python en utilisant des librairies standard telles que Twisted et ctypes. La gestion de l'historique d'un fichier est similaire à celle d'un gestionnaire de version classique dans le sens où il enregistre uniquement les différences entre deux versions successives d'un fichier, c'est le delta encoding.\\

Dropbox utilise le Amazon S3 pour ses serveurs.\\

\subsection{Critiques}

En juillet 2011, Dropbox a modifié ses conditions d'utilisation et peut maintenant utiliser les fichiers stockés par ses clients sans leur autorisation. Ceci a poussé de nombreux utilisateurs à abandonner le service et à s'orienter vers d'autres solutions ou l'auto-hébergement.\\

SparkleShare est une solution de stockage dans le cloud. Il est tres proche de Dropbox dans le fonctionnement une fois coffigurer. Il repose sur Git pour gerer la syncronisation des fichiers. Ce logiciel est libre (sous licence GPL3), compatible gnu/linux, windows, mac, ios et android. Il y a la possibilite de se connecter a n'importe quelle repertoire git dont github, gitorious, bitbucket.\\

La version 1.0 est sorti le 9 decembre 2012 mais le travail a commencer debut 2011 avec la premiere version publique le 14 fevrier 2011.\\

Le principal probleme de ce logiciel est d'etre base sur git et donc de souffrir des memes default que celui-ci. Par exemple, sur de gros fichiers git est lent. Par rapport a d'autre solution, il n'est pas compatible avec d'autre protocole (FTP/WebDAV/RSync pour citer que eux). Le logiciel ne fonctionne qu'avec un mode graphique, il y est donc impossible de l'utiliser le client en mode console. L'installation n'est pas des plus simple pour un utilisateur lamba qui veux utiliser sont propre serveur.\\

D'un cote technologie utilise, le logiciel se base sur git. Il est coder en C\# en utilisant mono.\\

\chapter{Google drive}
\thispagestyle{EIP} % seems mandatory

\section{Présentation}
Google Drive est une application web de stockage dans le cloud. Il s'agit d'une solution propriétaire accessible depuis un site internet ou différentes applications spécifiques à certaines plate-formes.

\section{Historique}
Google Documents est une application web créé en 2006, regroupant un tableur, un traitement de textes et un éditeur de diaporamas.
Lancé en 2012, Google Drive lui succède, proposant en plus un système de stockage en ligne, valable pour un grand nombre de types de fichiers.\\

\section{Description}
Ce service hérite des fonctionnalités de Google Documents, telles que le partage de données avec d'autres utilisateurs, mais aussi, par son interface, une édition simultanée d'un même document par plusieurs personnes (travail collaboratif).\\
Le principal apport de Google Drive est certainement la possibilité de synchroniser ses données avec des fichiers locaux, qui fait de lui un véritable service de cloud computing.\\
Il s'intègre parfaitement à d'autres applications Google comme Google+ ou Gmail.\\
\\
Il offre des capacités allant de 5Go à 16To, tout en permettant 10Go maximum par fichier, et possède des applications spécifiques pour les plate-formes Windows, Mac OS, Android, et prochainement iOS.\\

\section{Critiques}
Derrière une interface intuitive et agréable à l'utilisation, on retrouve un problème assez embêtant: les données, centralisées, échappent à l'utilisateur. Ce dernier n'a en effet aucun contrôle sur ce qu'il stocke, et le code fermé de l'application enpêche d'en porter une alternative sur un autre support.
\section{Windows Live Skydrive}
\thispagestyle{EIP} % seems mandatory

\subsection{Présentation}
Windows Live Skydrive est une solution de synchronisation et partage de fichiers. Développé par Microsoft, il peut s'utiliser à travers un navigateur internet ou encore des applications natives pour différents systèmes/

\subsection{Historique}
Windows Live Folders est lancé puis ouvert au grand public en août 2007. Il change de nom le même mois pour devenir Window Live Skydrive. Offrant au départ un espace de stockage gratuit de 5Go par utilisateur, il passe à 25Go en 2008 pour redescendre à 7Go en 2012.\\

\subsection{Description}
Il peut s'utiliser aussi bien en tant que service web qu'en tant qu'application lourde, permettant une synchronisation avec les données locales.\\
Un service de travail collaboratif est aussi disponible de façon à ce que différents utilisateurs puissent travailler sur les mêmes fichiers.\\
Il permet un accès à distance aux postes ayant installé le client Windows Live Skydrive.\\
\\
Ses capacités varient de 7 à 107Go, avec 2Go maximum par fichier.\\
Inclus à Office et Windows Phone, il possède des applications pour Windows, Mac OS, iOS et bientôt pour Android.\\

\subsection{Critiques}
Les critiques envers Google Drive peuvent aussi être énoncées dans le cas de Windows Live Skydrive. De plus, la synchronisation n'est au final disponible que pour un nombre limité de plate-formes, restreignant l'utilisateur.
\chapter{Owncloud}
\thispagestyle{EIP} % seems mandatory
\section{Présentation}
OwnCloud est une application web de stockage en ligne. Sous licence libre (AGPL), il peut-être installé sur n'importe quel serveur disposant de PHP et de SQL.\\
C'est une solution à installer soi-même, qui propose donc pas de louer un espace de stockage, mais d'en créer un.\\

\section{Historique}
Annoncé lors du Camp KDE 2010, le projet a bien évolué depuis. Le développement est très actif et suivi par la communauté (plus de 10000 commits et 1500 rapports de bogues en seulement 3 ans).\\

En 2011, une entreprise s'est créée autour du projet, proposant des services plus avancés pour les entreprises. Fin 2012, la société a réalisé une levée de fonds de 2,5 millions de dollards.

\section{Description}
OwnCloud repose sur un système d'applications. C'est donc un système modulaire, qui bénificie de très nombreuses fonctionnalités (édition de fichier, streaming de musique, synchronisation de calendrier et de contacts, gallerie photo, etc…).\\
Plusieurs clients sont disponible sur la plupart des plateformes, ainsi qu'une version web.\\

\section{Critiques}
Beaucoup d'avis d'utilisateurs critiquent les nombreux bogues d'OwnCloud, ainsi que la qualité de son code. Son plus gros avantage est qu'il permet d'installer rapidement une solution multi-usage, mais il semblerait qu'il se montre peu fiable sur le long terme.
Pour l'échange de fichiers, OwnCloud repose sur la technologie WebDAV, qui montre vite ses limites avec de gros fichiers.
À partir de la version 4.5, il est possible de s'interfacer avec Google Drive et Dropbox. Cette fonctionalité, encore expérimentale, et la celle qui se rapproche le plus du Cloud Computing. En effet, OwnCloud ne permet pas de répartir la charge sur plusieurs serveurs, notion inhérante au Cloud Computing.

\chapter{iCloud}
\thispagestyle{EIP} % seems mandatory

\section{Présentation}
Principalement utile pour les utilisateurs de produits Apple, iCloud permet une synchronisation, par cloud, entre les différents appareils qu'elle produit, il est par exemple fourni avec iOS.

\section{Historique}
Créée en 2011, cette application de la firme Apple regroupe iTunes in the Cloud, Photo stream, Calendar, Mail, et Contacts.

\section{Description}
Cette application permet la synchronisation de différents éléments (applications, livres, documents, sauvegardes) entre divers appareils. Il ne s'agit pas à proprement parler d'un service par navigateur, maisil dispose d'une interface web pour contrôler certaines de ses données.\\
Ses utilisateurs peuvent partager leurs fichiers à l'aide de la suite iWork, et bénéficient d'un utilitaire de sauvegarde automatique.\\
\\
Ils ont pour cela un espace allant de 5 à 55Go à leur disposition, ainsi qu'une taille maximale de 250Mo par fichier.\\
Il est intégré à iOS et propose des applications pour Windows et Mac OS.\\

\section{Critiques}
Cette application n'est disponible que pour très peu de systèmes, et l'interface web fortement limitée (accès uniquement aux données d'iWork). Pour en profiter pleinement, l'utilisateur doit ainsi s'orienter vers du tout Apple.
\section{Syncany}
\thispagestyle{EIP}
\subsection{Présentation}
Syncany is a storage application that is meant to be multi-purpose. It provides the possibility to save your data on very differents supports like FTP, IMAP, SSH, CIFS, Amazon S3... The software must permit to manage ones files in a very flexible and personal way.

\subsection{Historique}
This project was launched on april 27th 2011. This project was very followed at the beggining, before loosing his attractivity.

In october 2011, this project became a student project at the Mannheim University. From this moment, its public activity strongly reduced. By now, there is almost no communication as the last official message is from april 2012.

\subsection{Description}
Syncany wants to be a thick crossplatform application which would allow secure storage (local encrypted data) and would abstract storage system. So it bears a large number of different storage ways, in a transparent maner :

\begin{itemize}
\renewcommand{\labelitemi}{$\bullet$}
\item Local Folder
\item FTP
\item IMAP
\item Google Storage
\item Amazon S3
\item Rackspace Cloud Files
\item WebDAV
\item Picasa Web Albums
\item Windows Share (NetBIOS/CIFS)
\item Box.net
\item SFTP/SSH
\end{itemize}

\vspace{1cm}

Users can with only one interface store their files in their medium of choice. It is a very flexible solution because you are not obliged to use the software itself to access, add and edit files. It allows to create various workflows, to ensure that the storage medium will always be available and even after Syncany ceases to exist, files will always be reachable.

\subsection{Critiques}
Even if Syncany\'s offer seems to be very attractive, the leak of project-realted communication tends to asssume its death. And even if source code was available at the begining, it is not sufficent to be used in real situations.

As there is no official announcement of the project\'s death, it is difficult to know if it is possible to contribute or even resume the work already done.

This solution is, by now, more a proof of concept than a ended software.


\end{document}
