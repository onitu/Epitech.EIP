\documentclass[12pt]{report}

\usepackage[utf8]{inputenc}
\usepackage[french]{babel}
\usepackage{fullpage}
\usepackage{graphicx}
\usepackage{fancyhdr}	% headers/footers
\usepackage{xcolor}		% to use our own color
\usepackage{lastpage}	% to easily know the total number of pages
\usepackage{titling}	% to easily know the total number of pages
\usepackage{colortbl}	% to put color in a table background
\usepackage{datetime}	% to allow us set a new date formatting

% Define useful hand-made colors
\definecolor{epiBlue}{RGB}{23,54,93}
\definecolor{lightGray}{gray}{0.85}

% Bit of code to bold an entire table row
% http://tex.stackexchange.com/questions/4811/make-first-row-of-table-all-bold
\newcolumntype{$}{>{\global\let\currentrowstyle\relax}}
\newcolumntype{^}{>{\currentrowstyle}}
\newcommand{\rowstyle}[1]{\gdef\currentrowstyle{#1}%
  #1\ignorespaces
}

% Defining a "dd/mm/yyyy" date format
\newdateformat{dashDate}{\twodigit{\THEDAY}/\twodigit{\THEMONTH}/\twodigit{\THEYEAR}}

% fancyhdr-specific commands
\setlength{\headheight}{15.2pt}

%% Defining headers and footers contents.

% Big dirty hack of the "empty" pagestyle to show header and footer on the title page (in wait of a better solution)
\fancypagestyle{empty}
{
	\renewcommand{\headrulewidth}{0pt}
	\renewcommand{\footrulewidth}{1pt}
	\fancyhead[L]{\includegraphics[height=42pt]{logo_eip.png}
	}
	\fancyhead[R]{\colorbox{epiBlue}{\color{white}\textbf{\Large{[2015][TEMPLATE]} } }
	}
	
	\fancyfoot[L]{
   		\textcolor{gray}{\{EPITECH.\}}
	}
	\fancyfoot[C]{
		\jobname
	}
	\fancyfoot[R]{}
}

\fancypagestyle{EIP}
{
	\renewcommand{\headrulewidth}{0pt}
	\renewcommand{\footrulewidth}{1pt}
	\fancyhead[L]{\includegraphics[height=42pt]{logo_eip.png}
	}
	\fancyhead[R]{\colorbox{epiBlue}{\color{white}\textbf{\Large{[2015][TEMPLATE]} } }
	}
	
	\fancyfoot[L]{
   		\textcolor{gray}{\{EPITECH.\}}
	}
	\fancyfoot[C]{
		\jobname
	}
	\fancyfoot[R]{
		\thepage/\pageref{LastPage}
	}
}

\pagestyle{EIP} % does not seem to work ...

% end of fancyhdr stuff

%Gummi|063|=)

%\title{The Title\\\normalsize A Sub-title}
\title{
	\huge{\textbf{\textcolor{epiBlue}{Mon super titre} } }\\
	\Large{\textbf{\emph{\textcolor{gray}{Mon super sous-titre} } } }
}


\begin{document}
\addtocontents{toc}{\protect\refstepcounter{page}} % makes the table of contents count pages from 1 (one)
\maketitle

\thispagestyle{empty}
\vspace*{10mm}
\textbf{\emph{\textcolor{epiBlue}{Description du document} } }\\

\begin{tabular}{|>{\columncolor[gray]{0.85}\color{epiBlue} \bfseries } l|l|}
\hline
	Titre & Modèle document EIP\\
\hline
	Date & \dashDate\today \\
\hline
	Auteur & Alexandre BARON\\
\hline
	Responsable & Thérèse PONSABLE\\
\hline
	E-Mail & eip-tech@labeip.epitech.eu\\
\hline
	Sujet & Il faut sauver les bébés phoques\\
\hline
	Mots clés & Vachette, Bricard, Laperche\\
\hline
	Version du modèle & 1.0\\
\hline
\end{tabular}
\vspace*{10mm}

\textbf{\emph{\textcolor{epiBlue}{Tableau des révisions} } }\\


\begin{tabular}{|$l|^l|^l|^l|}
\hline
\rowcolor{lightGray}
\rowstyle{ \color{epiBlue} \bfseries}
	Date & Auteur & Section(s) & Commentaires\\
\hline
	\dashDate\today & Alexandre BARON & Toutes & Première version \\
\hline
	& & & \\
\hline
	& & & \\
\hline
\end{tabular}

\tableofcontents
\thispagestyle{empty}
\chapter{Foo}
\setcounter{page}{1} %reset the page count
\thispagestyle{EIP} % seems mandatory

\chapter{Bar}
\thispagestyle{EIP} % seems mandatory
%% \input{file.tex}
\section{Baz}


\end{document}
