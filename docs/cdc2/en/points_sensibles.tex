\section{Sensibles points}
Onitu has to be finished in two and a half years. During this period several sensible points must be payed particular attention to.\\

A first important point is Ubuntu One's protocol. Onitu might have to adapt quickly if Canoncical decides to change parts of its protocol. However such a decision is very unlickly and is unexpected.\\

Portability is another subject of special attention. This means we want to run as may tests as we can on a great number of platforms but that we might we obligated to change a library or functionality because it is not suported everywhere. IF a plateform cause to many troubles in isn't widly used we might consider dropping its support.\\

Onitu is confronted to the \textit{CAP} theorem, stating that in practical computer science it is impossible for a distributed computer system to simultaneously provide consitency, availability and partiion tolerance. Onitu's priority is preserving consitency and availability.\\

In general, development of a server is always a delicate matter that should not be under-estimated. Numerous test in different categories must we undertaken regulary and user feedback must be taken into account.
