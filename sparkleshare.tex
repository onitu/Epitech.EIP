SparkleShare est une solution de stockage dans le cloud. Il est tres proche de Dropbox dans le fonctionnement une fois coffigurer. Il repose sur Git pour gerer la syncronisation des fichiers. Ce logiciel est libre (sous licence GPL3), compatible gnu/linux, windows, mac, ios et android. Il y a la possibilite de se connecter a n'importe quelle repertoire git dont github, gitorious, bitbucket.\\

La version 1.0 est sorti le 9 decembre 2012 mais le travail a commencer debut 2011 avec la premiere version publique le 14 fevrier 2011.\\

Le principal probleme de ce logiciel est d'etre base sur git et donc de souffrir des memes default que celui-ci. Par exemple, sur de gros fichiers git est lent. Par rapport a d'autre solution, il n'est pas compatible avec d'autre protocole (FTP/WebDAV/RSync pour citer que eux). Le logiciel ne fonctionne qu'avec un mode graphique, il y est donc impossible de l'utiliser le client en mode console. L'installation n'est pas des plus simple pour un utilisateur lamba qui veux utiliser sont propre serveur.\\

D'un cote technologie utilise, le logiciel se base sur git. Il est coder en C\# en utilisant mono.\\
